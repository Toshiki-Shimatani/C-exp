\documentclass[a4paper,11pt]{jarticle}
% ファイル先頭から\begin{document}までの内容(プレアンブル)については,
% 教員からの指示がない限り, { } の中を書き換えるだけでよい.

% ToDo: 提出要領に従って,適切な余白を設定する
\usepackage[top=25truemm,  bottom=30truemm,
            left=25truemm, right=25truemm]{geometry}

% ToDo: 提出要領に従って,適切なタイトル・サブタイトルを設定する.
\title{プログラミング演習1レポート \\
       演習課題: 名簿管理プログラムの作成}

% ToDo: 自分自身の氏名と学生番号に書き換える
\author{氏名: 島谷 隼生 (Shimatani, Toshiki) \\
        学生番号: 09428526}

% ToDo: 教員の指示に従って適切に書き換える
\date{出題日: 2017年06月14日 \\
      提出日: 2017年07月26日 \\
      締切日: 2017年07月26日 \\}  % 注:最後の\\は不要に見えるが必要.

% ToDo: 図を入れる場合,以下の1行を有効にする
%\usepackage{graphicx}

\begin{document}
\maketitle

% 目次つきの表紙ページにする場合はコメントを外す
%{\footnotesize \tableofcontents \newpage}

%%%%%%%%%%%%%%%%%%%%%%%%%%%%%%%%%%%%%%%%%%%%%%%%%%%%%%%%%%%%%%%%
\section{概要}
%%%%%%%%%%%%%%%%%%%%%%%%%%%%%%%%%%%%%%%%%%%%%%%%%%%%%%%%%%%%%%%%

本演習では,外部からの入力データを計算機で扱える内部形式に変換して格納し,
それらを操作する方法について学習する.
具体的には,標準入力から与えられる名簿のCSVデータをC言語の構造体の配列に格納し,
それらをソートして表示するプログラムを作成する.

与えられたプログラムの基本仕様と要件,および,本レポートにおける実装の概要を以下に述べる.

\begin{enumerate}
\setlength{\parskip}{0mm}\setlength{\itemsep}{0mm}%この1行で箇条書きの行間を調整している
\item 基本仕様の概要
    \begin{enumerate}
    \item 標準入力から「ID,氏名,生年月日,住所,備考」からなる
          コンマ区切り形式 (CSV 形式) の名簿データを受け付けて,
          それらをメモリ中に登録する機能を持つ.CSV形式の例を以下に示す.

% 波括弧の内部 {...} だけ文字サイズ等を設定する書き方の例
% 以下の例はフォントサイズ:10pt 行送り:11pt
    {\fontsize{10pt}{11pt} \selectfont
        \begin{verbatim}
        09428900,Takahashi Kazuyuki,1977-04-27,Saitama,male
        09428901,Honma Mitsuru,1972-08-25,Hokkaidou,male
        09428902,Ogura Shinsuke,1976-07-23,Kanagawa,male
        09428903,Shibata saki,1968-03-16,Hyougo,female
        :
        \end{verbatim}
    }
    \item 標準入力から\%で始まるコマンドを受け付けて,
          ファイルを入出力したり,登録してあるデータを操作する機能を持つ.
          実装する方針のコマンドを表\ref{tbl:commands}に示す.
    \end{enumerate}
\item 機能や性能に関わる要件
    \begin{enumerate}
    \item 名簿データは配列などを用いて少なくとも$10000$件のデータを登録できるようにする.
          今回のプログラムでは,構造体struct profileの配列profile\_data\_store[]を宣言して,
          $10000$件のデータを格納できるようにする.
    \item 名簿データは構造体struct profileおよび構造体struct dataを利用して,
          構造を持ったデータとしてプログラム中に定義して利用する.
          実装すべきデータ構造は表\ref{tbl:structure_person}である.
          表中の$n$~bytesとは,$n$バイトのchar型配列を意味する.
    \item コマンドの実行結果出力に不必要なエラーや警告は標準エラー出力に出力する.
    \end{enumerate}
\end{enumerate}

\begin{table}[t] % 表の位置は原則として t または b である.hやHは使わない.
    \centering % この1行はbegin~endの中を中央寄せにする,というコマンド
    \caption{実装するコマンド}
    \label{tbl:commands}
    \begin{tabular}{|l|l|l|}
        \hline
        コマンド & 意味 & 備考\\
        \hline
        \verb|%Q(q)| & プログラムの終了(\verb|Quit|) & \\
        \hline
        \verb|%C(c)| & 登録項目数の出力(\verb|Check|) & \\
        \hline
        \verb|%P(p) n| & \verb|CSV| の先頭から\verb|n|番目を抜き出して表示(\verb|Print|) & \verb|n:0→全件,n:負→後から-n件|\\
        \hline
        \verb|%R(r) file| & \verb|file|から読込み(\verb|Read|) & \\
        \hline
        \verb|%W(w) file| & \verb|file|から書出し(\verb|Write|) & \\
        \hline
        \verb|%F(f) word| & 検索結果を表示(\verb|Find|) & \verb|%P|と同じ形式で表示\\
        \hline
        \verb|%S n| &  \verb|CSV| の\verb|n|番目の項目で整列(\verb|Sort|) & 表示はしない\\
        \hline
        
    \end{tabular}
\end{table}

\begin{table}[t]
\centering % この1行はbegin~endの中を中央寄せにする,というコマンド
    \caption{名簿データ}
    \label{tbl:structure_person}
    \begin{tabular}{|l|l|l|l|l|}
        \hline
        ID & 氏名 & 生年月日 & 住所 & 備考データ\\
        \hline
        $32$~bit整数 & $70$~bytes & \verb|struct date|
        & $70$~bytes & 任意長\\
        \hline
    \end{tabular}
\end{table}

また,本レポートでは以下の考察課題について考察をおこなった.

\begin{enumerate}
\setlength{\parskip}{2pt}\setlength{\itemsep}{2pt}%この1行で箇条書きの行間を調整している
    \item 不足機能についての考察
    \item エラー処理についての考察
    \item 新規コマンドの実装
    \item 既存コマンドの改良
    \item 構造体のサイズ
    \item 本課題の要件に対する考察
    \item コマンドの拡張
    \item テキスト形式とバイナリ形式
\end{enumerate}

%%%%%%%%%%%%%%%%%%%%%%%%%%%%%%%%%%%%%%%%%%%%%%%%%%%%%%%%%%%%%%%%
\section{プログラムの作成方針}
%%%%%%%%%%%%%%%%%%%%%%%%%%%%%%%%%%%%%%%%%%%%%%%%%%%%%%%%%%%%%%%%

プログラムをおおよそ以下の部分から構成することにした.
それぞれについて作成方針を立てる.

\begin{enumerate}
\setlength{\parskip}{2pt} \setlength{\itemsep}{2pt}
    \item 必要なデータ構造の宣言部(\ref{sec:declare}節)
    \item 標準入力から得た CSV データの解析部(\ref{sec:parse}節)
    \item 解析したデータの内部形式への変換部(\ref{sec:change}節)
    \item 各種コマンド実現部(\ref{sec:command}節)
\end{enumerate}


%--------------------------------------------------------------%
\subsection{宣言部} \label{sec:declare}
%--------------------------------------------------------------%

``宣言部''はプログラム中で使用する構造体,配列を宣言する部分である.
このレポートでは概要で示した表\ref{tbl:structure_person}に基づいて,
以下のように宣言する.

{\fontsize{10pt}{11pt} \selectfont
\begin{verbatim}
    struct date {
      int y;
      int m;
      int d;
    };

    struct profile {
      int id;
      char name[70];
      struct date birthday;
      char home[70];
      char *commment;
    };

    struct profile_data_store[10000];
\end{verbatim}
}

ここで,コメントについてはポインタを用いて宣言している.
これによって,長さを任意長にすることができる.

%--------------------------------------------------------------%
\subsection{解析部} \label{sec:parse}
%--------------------------------------------------------------%

``解析部''は標準入力から得られたデータの読み込みを行う箇所である.
しかし,このままでは,わかりにくい.
そこで,段階的詳細化の考え方に基づいてさらなる詳細化をおこない,
下記の(a)から(e)のように分割することにする.

\begin{enumerate}
\setlength{\parskip}{2pt} \setlength{\itemsep}{2pt}
\renewcommand{\labelenumi}{(\alph{enumi})} % この1行はリスト見出しを(a), (b)に変えるためのコマンド
    \item EOFが来るまで,標準入力から文字の配列\verb|char line[]|に1行分を読み込む.
    \item \verb|line|の1文字目が,\verb|'%'|ならば,2文字目をコマンド名,4文字目以降をその引数として,決定されたコマンドを実行する.
    \item さもなくば\verb|line|をCSVとみなし\verb|','|を区切りとして5つの文字列に分割する.
    \item 分割してできた5つの文字列を変換部に渡し構造体に代入する.
    \item 次の行を読み込む
\end{enumerate}

ここで,\verb|(b)|は``コマンド文字に応じて分岐''と詳細化することもできるが,今回はそのままとしておく.また,\verb|(d)|で扱う文字列はchar型配列として処理するため,解析部に続く変換では型変換に注意する必要がある.

%--------------------------------------------------------------%
\subsection{変換部} \label{sec:change}
%--------------------------------------------------------------%

``変換部''は分割されたCSVデータを項目毎に型変換し,
対応する構造体メンバに保存する部分である.
メンバに様々な型を用いているため,適切な代入の使い分けが必要となる.

文字列は関数\verb|new_profile|を用いて代入する.
数値の場合,関数\verb|atoi|を用いて文字列を数値化してから代入する.
構造体\verb|struct date|であるメンバ\verb|birthday|については更に分割してから代入する.

なお,構造体への代入については,関数\verb|split|を用いることで容易に実装することができる.
例えば,\verb|"2014-10-25"|のような文字列を読み込んで,
\verb|'-'|にごとに分割しつつ格納するという処理は,csvデータを分割しつつ格納する処理と同じ処理である.
従って,区切り文字がCSVの\verb|','|とは異なる\verb|'-'|になること以外は同様に記述できるはずである.


%--------------------------------------------------------------%
\subsection{各種コマンド実現部} \label{sec:command}
%--------------------------------------------------------------%

``各種コマンド実現部''は構造体の配列に格納したデータに対する処理をおこなう部分である.
このレポートでは,具体的には\verb| %Q(Quit),%C(Check),%P(Print),%R(Read),%W(Write),%F(Find),%S(Sort)|,その他の追加コマンドを実装している.また,コマンドの文字については,
大文字と小文字の双方で反応する仕様としている.

終了(\verb|%Q|)は\verb|exit|関数を用いてプログラムを終了させればよい.

チェック(\verb|%C|)は\verb|new_profile|関数中でカウントされた\verb|profile_data_ntimes|を\verb|printf|関数で表示すればよい.

表示(\verb|%P n|)は\verb|printf|関数で各項目毎に表示すればよい.
ただし,\verb|n|が負であるとき,後ろから\verb|-n|件表示することに注意が必要である.

読み込み(\verb|%R file|)は\verb|get_line|関数を拡張して引数にファイルポインタを加えた上でで,main関数中の処理にファイルの開閉処理を加えることで実装できる.このとき,拡張前後でプログラムに支障がないかテストが必要である.

書き込み(\verb|%W file|)は\verb|%P|コマンドに手を加えることで実装可能である.具体的には,\verb|%P|コマンド中の範囲指定を取り除き,出力の仕方をCSV形式に調整し,ファイルの開閉処理を加えればよい.

検索(\verb|%F word|)も\verb|%P|コマンドに手を加えることで実装可能である.\verb|%W|コマンドと同様に範囲指定は取り除くが,今度は入力(\verb|word|)された文字列と構造体メンバが一致するか調べる必要がある.文字列,数,構造体のそれぞれを入力文字列と比較するが,今回は数と構造体をともに文字列に変換する方針をとる.その詳細は実装過程における考察において述べる.

ソート(\verb|%S n|)は今回,最も単純なバブルソートを用いることにする.引数の\verb|n|の値によってソートの判定に用いる項目が異なるが,二要素間を入れ替えるかどうかを判定する\verb|compare_profile|関数中にてswitch文を用いることで解決できる.また,入れ替えを行うときはポインタで操作することに注意する.

その他の追加コマンドの方針等ついては$6.3$節にて説明する.

%%%%%%%%%%%%%%%%%%%%%%%%%%%%%%%%%%%%%%%%%%%%%%%%%%%%%%%%%%%%%%%%
\section{プログラムおよびその説明}
%%%%%%%%%%%%%%%%%%%%%%%%%%%%%%%%%%%%%%%%%%%%%%%%%%%%%%%%%%%%%%%%

プログラムリストは$8$節に添付している.プログラムは全部で$573$行からなる.
以下では,前節の作成方針における分類に基づいて,プログラムの主な構造について説明する.

%--------------------------------------------------------------%
\subsection{宣言部(11行目から56行目)}
%--------------------------------------------------------------%

前節で示したように,宣言部ではプログラム中で使用する構造体や配列,加えてグローバル変数を宣言する.
\verb|date|構造体や\verb|profile|構造体はデータに関する個別の情報をそれぞれ格納するために,配列\verb|profile_data_store|はデータ全体を格納するために用いる.グローバル変数\verb|profile_data_nitems|は現在までに何件のデータを格納しているかを各関数で共有するために用いる.\verb|save|はデータをファイルに保存したかどうかを示す.詳しくは,$5.4$節で説明する.
また,今後使う関数も28--56行目でプロトタイプ宣言をしておく.

%--------------------------------------------------------------%
\subsection{解析部(56行目から175行目)}
%--------------------------------------------------------------%

49--55行目は\verb|main|関数であり,作成方針で説明した解析部の動作におおよそ相当する.
ただし (c) の 5つの文字列に分割する部分は,解析部の\verb|main()|関数では実現せず,
変換部である\verb|new_profile|関数中で\verb|split|を呼出すことにしている.
57--111行目は\verb|main|関数内で用いる関数とそれに関係する関数郡である.
(a) は \verb|get_line|関数, (b) は\verb|parse_line|関数,
\verb|exec_command|関数によって実現している.

解析部の関数に用いているので,文字列操作関数\verb|subst()|関数をこの部分に含めている.
\verb|subst()|関数は78--96行目で宣言している.       

\verb|subst| は,\verb|str|が指す文字列中の\verb|c1|文字を\verb|c2|に置き換える.
プログラム中では,入力文字列中の末尾に付く
改行文字をヌル文字で置き換えるために使用している.

%--------------------------------------------------------------%
\subsection{変換部(177行目から240行目)}
%--------------------------------------------------------------%

177--223行目は変換部の主となる\verb|new_profile|関数である.
225--237行目は\verb|split|関数である.解析部ではなく変換部で呼び出した理由については,
考察にて後述する.

\verb|split| は\verb|str|が指す文字列を区切文字\verb|c|で分割し,
分割した各々の文字列を指す複数のポインタからなる配列を返す関数である.
プログラム中では,CSVを\verb|','|で分割し,
分割後の各文字列を返すのに使用されている.
また,``2004-05-10'' のような日付を表す文字列を `-' で分割して,
struct date を生成する際にも使用している.

%--------------------------------------------------------------%
\subsection{コマンド実現部(242行目から576行目)}
%--------------------------------------------------------------%

242行目--469行目は各種コマンド関数であり,472行目--576行目は各種コマンド関数に用いる
関数郡である.できる限り一つの関数に複数の処理が重ならないように意識した.
特にソートに関する関数は条件分岐が多くなったので,条件部分も関数として作成した.
コマンドに関する説明は次節にて行う.
%%%%%%%%%%%%%%%%%%%%%%%%%%%%%%%%%%%%%%%%%%%%%%%%%%%%%%%%%%%%%%%%
\section{プログラムの使用法}
%%%%%%%%%%%%%%%%%%%%%%%%%%%%%%%%%%%%%%%%%%%%%%%%%%%%%%%%%%%%%%%%

本プログラムは名簿データを管理するためのプログラムである.
CSV形式のデータと \% で始まるコマンドを標準入力から受け付け,
処理結果を標準出力または\verb|%W(w)|コマンド,\verb|%BW(bw)|コマンドで指定されたファイルに出力する.
入力形式の詳細については,概要の節を参照のこと.

プログラムは,一般的な UNIX で用いることを意図している 
\verb|gcc|でコンパイルした後,標準入力からファイルを与える,
もしくは手入力でCSV形式でデータを入力していく.

{\fontsize{10pt}{11pt} \selectfont
 \begin{verbatim}
   \$ gcc -o program1 program1.c
   \$ ./program1 < csvdata.csv

   \$ gcc -o program1 program1.c
   \$ ./program1
   \$ 09428900,Takahashi Kazuyuki,1977-04-27,3,Saitama,male
 \end{verbatim}
}

プログラムの出力結果としてはCSVデータの各項目を読みやすい形式で出力する.
例えば,下記の cvsdata.csv に対して,

{\fontsize{10pt}{11pt} \selectfont
 \begin{verbatim}
  09428900,Takahashi Kazuyuki,1977-04-27,Saitama,male
  09428901,Honma Mitsuru,1972-08-25,Hokkaidou,male
  %C
  %P
  %P 1
  %Q
 \end{verbatim}
}

\noindent 
以下のような出力を得る.

{\fontsize{10pt}{11pt} \selectfont
 \begin{verbatim}
  2 profile(s)
   
  Id    : 94285900
  Name  : Takahashi Kazuyuki
  Birth : 1977-04-27
  Addr  : Saitama
  Com.  : male

  Id    : 94285901
  Name  : Honma Mitsuru
  Birth : 1972-08-25
  Addr  : Hokkaido
  Com.  : male
  
  Id    : 94285900
  Name  : Takahashi Kazuyuki
  Birth : 1977-04-27
  Addr  : Saitama
  Com.  : male
 \end{verbatim}
}

\noindent
入力された
\verb|%C|は,これまでの入力データが
何件登録されたかということを示し,
\verb|%P| は 入力したデータを全件表示することを
示している.また,
\verb|%P 1|は入力したデータを先頭から
1件(負の数だと後ろから)表示することを示し,
\verb|%Q|はプログラムを終了することを示す.

上の例では\verb|%Q|,\verb|%C|,\verb|%P|コマンドの説明を行った.
以下では基本方針の残りの\verb|%R|,\verb|%W|,\verb|%F|,\verb|%S|コマンドの説明を行う.
以下のようなsample.csv,sample2.csvに対して,

{\fontsize{10pt}{11pt} \selectfont
 \begin{verbatim}
(sample.csv)
   %R sample2.csv
(sample2.csv)
   %C
   09428900,Takahashi Kazuyuki,1977-04-27,Saitama,male
   09428901,Honma Mitsuru,1972-08-25,Hokkaidou,male
   %C
   09428902,Nakamura Hiroki,1975-09-04,Nagano,male
   %W sample.csv
   %S 3
   %P  
   %F Nagano
  \end{verbatim}
}
\noindent
以下のような出力を得る.
{\fontsize{10pt}{11pt} \selectfont
 \begin{verbatim}
   0 profile(s)  
   2 profile(s)
  
   Id    : 94285901
   Name  : Honma Mitsuru
   Birth : 1972-08-25
   Addr  : Hokkaido
   Com.  : male

   Id    : 94285902
   Name  : Nakamura Hiroki
   Birth : 1975-09-04
   Addr  : Nagano
   Com.  : male
  
   Id    : 94285900
   Name  : Takahashi Kazuyuki
   Birth : 1977-04-27
   Addr  : Saitama
   Com.  : male  
   
   Id    : 94285902
   Name  : Nakamura Hiroki
   Birth : 1975-09-04
   Addr  : Nagano
   Com.  : male  
 \end{verbatim}
}
\noindent
出力後,sample.csvは以下のように書き換えられている.

{\fontsize{10pt}{11pt} \selectfont
 \begin{verbatim}
   09428900,Takahashi Kazuyuki,1977-04-27,Saitama,male
   09428901,Honma Mitsuru,1972-08-25,Hokkaidou,male
   09428902,Nakamura Hiroki,1975-09-04,Nagano,male
  \end{verbatim}
}
\noindent
入力された\verb|%R sample2.csv|は指定されたファイル(この場合,sample2.csv)を
読み込むことを示し,\verb|%W sample.csv|は指定されたファイル(この場合,sample.csv)
に書き込みをすることを示す.また,\verb|%S 3|はこれまでの入力データを
3番目の項目(生年月日)でソート(バブルソート)することを示しており,
\verb|%F Nagano|は入力されたワード(この場合,Nagano)と一致する項目をもつデータを
表示する.これら以外にもコマンドはあるが,それらについては$6.3$節で説明する.

%%%%%%%%%%%%%%%%%%%%%%%%%%%%%%%%%%%%%%%%%%%%%%%%%%%%%%%%%%%%%%%%
\section{作成過程における考察}
%%%%%%%%%%%%%%%%%%%%%%%%%%%%%%%%%%%%%%%%%%%%%%%%%%%%%%%%%%%%%%%%


2節で述べた実装方針に基づいて,3節ではその実装をおこなった.
しかし,実装にあたっては実装方針の再検討が必要になる場合があった.
本節では,名簿管理プログラムの作成過程において検討した内容,
および,考察した内容について述べる.

%--------------------------------------------------------------%
\subsection{宣言部についての考察}
%--------------------------------------------------------------%

宣言部については,概ね方針通りに実装することができた.また,3節でも述べたようにグローバル変数を使用した.
これは,グローバル変数を使用することで処理するデータを関数間で共有することが容易にできるためである.ただ,グローバル変数は他人がプログラムに変更を加えるときの障害になる可能性を持つことを留意しておく必要がある.今回は,自分以外の人が手を加えることを想定していないため,遠慮なく用いている.
また,\verb|profile|構造体のメンバ\verb|id|に8桁の制限をかけている.これは9桁となるとint型の範囲を超えてしまうためである.
実際に制限をかけているプログラムは変換部の\verb|new_profile|関数中にて実装している.
%--------------------------------------------------------------%
\subsection{解析部についての考察}
%--------------------------------------------------------------%

解析部の作成方針として読み込んだ行がコマンドでなければ5つの文字列に分解するという方針としたため,
分解後の文字列が数字であろうとなかろうと,char型配列であることに注意が必要である.なぜなら宣言部にて宣言した構造体のメンバに,
int型のメンバがものが含まれているため,続く変換部にて構造体のそのまま代入ができない.これを解決するにはキャストすることも考えられるが,
今回は\verb|atoi|関数を用いることで対処した.なぜならば,キャストした文字列は数字であることを想定しているが,
もし文字であった場合の挙動が予測できないので,文字であった場合の動作が保証されている\verb|atoi|関数を選択した.

%--------------------------------------------------------------%
\subsection{変換部についての考察}
%--------------------------------------------------------------%

変換部についても,概ね方針通りに実装することができた.だが,入力されたCSV形式のデータの要素数(分割数)が少ない場合,
その入力を不正なものとして処理して無効な入力としている.これについては,後のエラー処理に関する考察で説明する.
また,\verb|date|構造体に格納するときの分割処理においては,
要素数(分割数)が規定の値(今回は年,月,日のため3)出ない場合,同様のエラー処理を行っている.
このため,正しい入力の仕方をしなければ,延々とエラー出力が続くようになっている.
そのため,標準エラー出力によって正しい入力の仕方を表示するようにしている.
また,変換部にて文字列分割処理を行うこととした理由は,このような方針とすることで\verb|new_profile|関数に渡す引数が,
1つの文字列のみでよくなり,関数内の処理を行いやすくできるためである.

%--------------------------------------------------------------%
\subsection{コマンド実現部についての考察}
%--------------------------------------------------------------%

コマンド実現部についても,概ね方針通りに実装した.\verb|%F|コマンドについては入力された文字列と構造体メンバが
一致しているかどうかの判定に\verb|strcmp|関数を使用しているため,
部分一致の場合は検索に引っかからない.今回はサンプルプログラムと同様の挙動を意識したため,
このような仕様となっているが,部分一致の場合も反応するためには独自の比較関数を作成して,
比較する実装方針とするほうがよいだろう.
また,\verb|%F|コマンドの比較操作部分で登録されているデータのメンバをすべて文字列へと変換する方針とした理由は,
入力されたデータを変化させることなく続けて各メンバとの比較を行うことができるからである.
また,入力データを各メンバに合わせて変換する方針とした場合,
入力されたデータがどのメンバと比較したいのかを判断できなければ型変換が困難となることも理由の一つである.
また,\verb|%Q|コマンドについてであるが,データをファイルに保存(書き込み)する前に,
誤って終了してしまう可能性も考えられる.そのため,ファイルに書き込むコマンドを使用していない場合,本当にプログラムを
終了してよいかどうかを確認するようにしている.グローバル変数で宣言した\verb|save|はファイルに書き込むコマンドを
使用したかどうかを$0,1$で判断するために使用している.

%%%%%%%%%%%%%%%%%%%%%%%%%%%%%%%%%%%%%%%%%%%%%%%%%%%%%%%%%%%%%%%
\section{結果に関する考察}
%%%%%%%%%%%%%%%%%%%%%%%%%%%%%%%%%%%%%%%%%%%%%%%%%%%%%%%%%%%%%%%%

演習課題のプログラムについて仕様と要件をいずれも満たしていることを
プログラムの説明および使用法における実行結果例によって示した.
ここでは,概要で挙げた以下の項目について考察を述べる.

\begin{enumerate}
\setlength{\parskip}{2pt} \setlength{\itemsep}{2pt}
    \item 不足機能についての考察
    \item エラー処理についての考察
    \item 新規コマンドの実装
    \item 既存コマンドの改良
    \item 構造体のサイズ
    \item 本課題の要件に対する考察
    \item コマンドの拡張
    \item テキスト形式とバイナリ形式
\end{enumerate}

%--------------------------------------------------------------%
\subsection{不足機能についての考察}
%--------------------------------------------------------------%

考えられる不足機能としては,部分一致でも反応する新たな\verb|%F|コマンド,登録データの削除機能,修正機能,
実装してあるコマンドを説明する機能などが考えられる.
それぞれの理由を説明する.まず,部分一致でも反応する新たな\verb|%F|コマンドについてだが,今回実装した\verb|%F|コマンドでは完全一致させる必要があり,不便である.また,探したいデータについての情報が不確かな場合に見つけることが困難になるという問題点もある.
次に,登録データの削除機能についてである.本プログラムでは,不正な入力は登録されないような仕様としたが,正しい入力形式で入力されたデータの要素に
間違いがあった場合はファイル書き出しを行い,プログラム終了後にemacsなどを用いて書き出したファイルの編集を行って対処しなければならない.
これでは大変不便である.そのため,プログラム実行中に削除できる機能が必要であると考えられる.修正機能についても同様である.
また,削除,修正するデータを指定するために,該当するデータが先頭から何番目かを調べる機能も必要であると考えられる.
実装してあるコマンドを説明する機能は,いわゆるヘルプ機能である.プログラムを用いる人がプログラムの使用方法を完全に知っている
可能性は高くないため,それを補う必要がある.そのための機能である.

%--------------------------------------------------------------%
\subsection{エラー処理についての考察}
%--------------------------------------------------------------%

\subsubsection{CSVデータ処理中のエラー処理}

CSVデータ中に,不正なデータが含まれていた場合の処理について考察する.
エラーが含まれていた場合は,以下のような対処が考えられる.

\begin{description} % descriptionで見出し後に開業するときは ~\\ とする.
  \item[(1) エラーのあった行を指摘して,無視する]~\\
    この方法は,一回の入力で,できるだけ多くの
    エラーを発見できるため,通常はこの方法が好ましい.
    しかし,エラーのあった状態からの復帰を行う必要があるため
    プログラムが複雑になる.
  \item[(2) エラーのあった行を指摘して,終了する]~\\
    この方法は,入力中に 1つのエラーを発見することしかできない.
    しかし,エラーのあった入力をデータを無視してしまうと
    以降のデータ入力の正当性チェックにも影響がでるような場合には,
    この方法を採らざるを得ないこともある.
\end{description}

エラーのあった行を指摘せず,終了または無視するという
方法も考えられるが,正常終了との区別が付かないため実用的でない.

今回は,エラーのあった行を指摘して,無視する方法がよいと考えた.理由は,ファイル読み込みなどで大量のCSVデータを読み込む際に,一つのエラー入力で
それ以降の入力が中止されてしまうと,データの登録にかかる時間が大幅に増えてしまう.そうであるならば,誤った入力を無視してしまったほうが効率は良くなる.
ただ,その場合はファイル読み込みの際にエラーが何行目で発生したかを表示させるべきと考えられる.

\subsubsection{データを保存する前にプログラムを終了してしまう際のエラー処理}
ユーザが登録した名簿データをファイルに保存(\verb|%W|コマンドを使用)する前にプログラムを終了してしまう可能性は十分に考えられる.
その対処としては,保存を行っていない場合でプログラムを終了しようとする際に,\verb|%W|コマンド,\verb|%BW|を使用していないという警告をだすことや,一時保存ファイルを作成することが考えられる.どちらが適しているか考えると,警告を出してユーザに確認をとる方がユーザに気づかせることもできることから,前者の方がてきしているだろう.今回は\verb|%Q|コマンドでそれを
実装している.

%--------------------------------------------------------------%
\subsection{新規コマンドの実装}
%--------------------------------------------------------------%
$6.1$節を踏まえて,新規コマンドとして登録されたデータを修正をコマンド\verb|%M n|を実装した.これはメモリ中のデータを指定し,CSV形式でデータを入力し直す
ことでデータの修正を行うものである.引数\verb|n|は先頭から\verb|n|番目のデータを修正することを意味している.\verb|n|が登録件数よりも大きい場合,
または負の数である場合は,エラー出力を行う.
また,登録されたデータを削除するコマンド\verb|%D n|も実装した.引数に関する説明や仕様は\verb|%M n|コマンドと同様である.
また,登録されているデータが先頭から何番目かを調べる機能については,\verb|%F|コマンド中の処理でカウントされているためそれを利用して
新たに\verb|%f|コマンドを実装した.機能は\verb|%F|コマンドと全く変わらないが,検索したデータが先頭から何番目かを表示する機能を追加している.また,詳細は$6.8$節にて述べるが,データをバイナリ形式でファイルに入出力するコマンド\verb|%BW word,%BR word|も実装している.

%--------------------------------------------------------------%
\subsection{既存コマンドの改良}
%--------------------------------------------------------------%
\verb|%W|コマンドは利用する際に毎回ファイル名を指定する必要があり,不便である.改良案として,同じファイルに続けて書き込みを行う際には引数に'/'を
用いることで,同じファイル名を繰り返し入力することを避ける方法が挙げられる.'/'とした理由は,'/'はlinuxにおいてファイル名として使用できない文字であるから,
ファイル名を引数とする\verb|%W|コマンドに適していると考えたからである.
また\verb|%S|コマンドのソートの方法として,本プログラムではバブルソートを用いているが,この方法は効率が悪いと言われている.よって,
より効率化をはかるためには,クイックソートを用いることが挙げられる.さらに,クイックソートの軸を選ぶ際にいくつかの要素(3つ程度)をとり,その平均を軸としたり,
ソートが行われていない残りの要素数が少なくなったときに挿入法に切り替えることでさらなる効率化をはかることができる.
今回はそういった工夫は行わなかったが,ソートコマンドの改良として,クイックソートを実装した.\verb|%S|と大文字としたときは
バブルソートを,\verb|%s|と小文字にしたときはクイックソートを実行するようにしている.
また,$5.4$節でも述べたが,\verb|%Q|コマンドに関する改良(ファイルを書き込みしていないときの確認)を行っている.
%--------------------------------------------------------------%
\subsection{構造体のサイズ}
%--------------------------------------------------------------%
以下のプログラムをgccでコンパイルし,実行した.

{\fontsize{10pt}{11pt} \selectfont
 \begin{verbatim}
         struct profile a;
        printf("size1..%d\n",sizeof(a));
        printf("size2..%d\n",sizeof(a.id));
        printf("size3..%d\n",sizeof(a.name));
        printf("size4..%d\n",sizeof(a.birthday));
        printf("size5..%d\n",sizeof(a.home));
        printf("size6..%d\n",sizeof(a.comment));
        printf("1..%p\n",&a);
        printf("2..%p\n",&a.id);
        printf("3..%p\n",&a.name);
        printf("4..%p\n",&a.birthday);
        printf("5..%p\n",&a.home);
        printf("6..%p\n",&a.comment);
 \end{verbatim}
}
\noindent
以下の結果を得た.

{\fontsize{10pt}{11pt} \selectfont
 \begin{verbatim}
        size1..164
        size2..4
        size3..70
        size4..12
        size5..70
        size6..4
        1..0xbfeff84c
        2..0xbfeff84c
        3..0xbfeff850
        4..0xbfeff898
        5..0xbfeff8a4
        6..0xbfeff8ec
 \end{verbatim}
}

これを見ると構造体のメンバのサイズの合計が160,構造体のサイズが164となっており一致していない.それぞれのアドレスを確認してみると,
メンバnameとメンバbirthday,メンバhomeとメンバcommentの間に隙間が空いていることが確認できる.これはハードウェア(CPU)の都合によって,
型によっては配置できるアドレスに制限があることや,もしくは配置することができても効率が悪くなるようなCPUがあり.そのような場合に,コンパイラが適当に境界調整
(アラインメント)行い,構造体に適切なパティング(詰め物)が挿入されるからである.また,アラインメントが構造体の末尾に入ることもあり、
そのような場合は構造体にsizeof演算子を適用すると,末尾のアラインメントを含めたサイズが返される.このため,構造体のメンバのサイズの合計と構造体のサイズが
異なるのであろうと考えられる.

%--------------------------------------------------------------%
\subsection{本課題の要件に対する考察}
%--------------------------------------------------------------%

本課題の要件のうち,コマンドが1文字であるのは,直感的ではない上に,コマンドの数が増えた場合の拡張性に乏しい.また,コマンドの後のスペースが
1個のみしか許されないというのは制限が強い.しかしプログラムの作成が容易になるというメリットがある.
また,課題ではコマンドはすべて大文字で記載されていたが,それにこだわる必要性は薄いと考えたため,小文字でも,大文字と同様の機能を行える仕様としている.
また,学校名や所在地が70byteまでとした理由が不明である,10000件ものデータを格納できるようにと定めているのであるから,データは節約すべきだろう.
少なくとも学校名は40--50byteでも問題はないだろう.

%--------------------------------------------------------------%
\subsection{コマンドの拡張}
%--------------------------------------------------------------%

1文字だけではなく,2文字のコマンドを受け付ける機能を実装するには,まず,1文字コマンドか,2文字コマンドのどちらかを
判断するが必要がある.これを実装するには2つの方法が考えられる.1つ目は読み込んだ入力文字列を\verb|split|関数を用いて,
分解し,分解された文字列の最初の文字列の文字数で1文字か2文字を判定する方法,2つ目は,読み込んだ入力文字列の2文字目と
4文字目以降の文字列だけではなく,3文字目も読み込み,それが空白文字またはナル文字か否かで判定する方法である.
今回は,後者を選択した.理由としては,後者の方がより少ない処理でコマンドの文字数判定が行えるからである.

実装の手順は,まず\verb|exec_command|関数へ渡す引数を拡張して3文字目を渡すようにし,
条件文を用いて,3文字目が空白かナル文字ならば1文字コマンド,それ以外ならば2文字コマンドを実行するようにする.
2文字コマンドを実行する手段としては,\verb|switch|文を二重で使った.これにより,\verb|switch|文を同様に増やすこと
で新たな2文字コマンドが実装しやすくなっている.その代わり,\verb|exec_command|関数が長くなるという欠点がある.
また,1文字コマンドのときは,4文字目以降の文字列を引数として渡していたが,2文字コマンドの場合は$+1$をして,
5文字目以降の文字列を渡す必要があることに注意する.

実際に動作するかの確認として\verb|%R|コマンドと同じ機能を持つ\verb|%TR|コマンドを実装した.結果は問題なく動作した.
また,このコマンドは\verb|%R|コマンドを流用しているので,新規コマンドには加えていない.

%--------------------------------------------------------------%
\subsection{テキスト形式とバイナリ形式}
%--------------------------------------------------------------%
バイナリ形式とは,テキスト形式と対比される用語で,CPUがそのまま扱えるデータ形式のことである.また,これはテキスト形式と
異なり,人間がそのまま読めない形式となっている.さらに,バイナリ形式でデータを保存する場合,効率が良くなる場合がある.
例えば,'112'というデータを扱う場合,バイナリ形式で扱えば'70'の1バイト分で済むこととなる.ただ,これは整数を
多く取り扱う場合に限ってであり,可変長である文字列が多い場合,テキスト形式の方が効率がよくなる.今回は,可変長である文字列は
commentのみであるため,バイナリ形式で効率がよくなる可能性は小さくない.

c言語を用いてデータをバイナリ形式でファイルに入出力するに当たって,使用できる関数は\verb|fwrite|関数,
\verb|fread|関数の2つのみである.今回は,この2つを主として実装を行っていった.実装していく中で大きな問題点と
なったことが,commentがポインタであることである.ポインタを含む構造体を出力し,次に読み込んだ時,そのポインタが
利用できないからである.もし利用しようとすれば,すぐにSEGVとなる.そのため,利用する際はポインタを初期化しなければ
ならず,その際ポインタ内のデータは読み込めない.そのため,可変長文字列をバイナリ形式のファイルを通じてやりとり
することはできないだろう(もしかしたらそれを実現する方法があるかも知れないが,見つけ出すこと,考え出すことはできなかった).
そのため,commentを要素数の決まった(今回は講義で配布されたsample.csvを参考に要素数を100とした)配列を用いて,
\verb|strncpy|関数を用いてバイナリ形式ファイルに出力した.そのため,commentが100文字に制限されてしまっている.
また,\verb|for|文を用いてループさせて,すべてのデータを出力させているが,ループごとに配列を初期化する必要がある.
そうしなければ,前回のループ時の要素数が今回のループ時の要素数より多かった場合に,不正な値が出力されてしまう.
バイナリ形式ファイルから入力を得るには,出力と逆の手順をすればよい.
%%%%%%%%%%%%%%%%%%%%%%%%%%%%%%%%%%%%%%%%%%%%%%%%%%%%%%%%%%%%%%%%

%%%%%%%%%%%%%%%%%%%%%%%%%%%%%%%%%%%%%%%%%%%%%%%%%%%%%%%%%%%%%%%%
%%%%%%%%%%%%%%%%%%%%%%%%%%%%%%%%%%%%%%%%%%%%%%%%%%%%%%%%%%%%%%%%
\section{作成したプログラム}
%%%%%%%%%%%%%%%%%%%%%%%%%%%%%%%%%%%%%%%%%%%%%%%%%%%%%%%%%%%%%%%%

作成したプログラムを以下に添付する.
与えられた課題については,節で示したようにすべて正常に動作したことを付記しておく.
{\fontsize{10pt}{11pt} \selectfont
\begin{verbatim}
     1  #include <stdio.h>
     2  #include <stdlib.h>
     3  #include <string.h>
     4  
     5  #define MAX_LINE_LEN 1024 /*1行に読み込める最大文字数*/
     6  #define MAX_STR_LEN 70 /*構造体メンバnameとhomeの最大文字数*/
     7  #define MAX 5 /*split関数での最大分割数*/
     8  #define PDS profile_data_store
     9  #define PDN profile_data_nitems
    10  
    11  struct date {
    12    int y;
    13    int m;
    14    int d;
    15  };
    16  
    17  struct profile {
    18    int id;
    19    char name[70];
    20    struct date birthday;
    21    char home[70];
    22    char *comment;
    23  };
    24  
    25  struct profile profile_data_store[10000]; /*10000件のデータまで登録可能*/
    26  int profile_data_nitems = 0; /*登録したデータ数*/
    27  int save = 0; /* データを保存したかどうかの判定に使用 */
    28  /*************************************************************************/
    29  int get_line(FILE *fp,char *line);
    30  int subst(char *str, char c1, char c2);
    31  void parse_line(char *line);     
    32  void exec_command(char cmd1,char cmd2, char *param);
    33  void new_profile(char *line);
    34  int split(char *str, char *ret[], char sep, int max);
    35  void cmd_quit();                 /************************************/ 
    36  void cmd_check();
    37  void cmd_print(int n);
    38  void cmd_read(char *file);
    39  void cmd_write(char *file);      /*             コマンド関数           */
    40  void cmd_find(char *word);
    41  void cmd_bsort(int n);
    42  void cmd_qsort(int n);
    43  void cmd_modify(int n);
    44  void cmd_delete(int n);
    45  void cmd_find2(char *word);
    46  void cmd_Bwrite(char *line);
    47  void cmd_Bread(char *line);     /************************************/
    48  void print_profile(struct profile *p);
    49  void fprint_profile_csv(FILE *fp, struct profile *p);
    50  char *date_to_string(char buf[], struct date *date);
    51  int comparsion (char *word, int i);/*        コマンド関数に用いる関数     */
    52  void b_sort(struct profile *p, int left, int right, int column);
    53  int compare_profile(struct profile *p1, struct profile *p2, int column);
    54  void swap (struct profile *a, struct profile *b);
    55  void q_sort(struct profile *p, int left, int right, int column);
    56  /*************************************************************************/
    57  int main() {
    58    char line[MAX_LINE_LEN + 1];
    59    while (get_line(stdin, line)) {
    60      parse_line(line);
    61     }
    62    cmd_quit();
    63    return 0;
    64  }
    65  
    66  int get_line(FILE *fp, char *line)
    67  {
    68    if (fgets(line, MAX_LINE_LEN + 1, fp) == '\0')
    69      {
    70        return 0;
    71      }
    72  
    73    subst(line, '\n', '\0');
    74  
    75    return 1;
    76  }
    77  
    78  int subst(char *str, char c1, char c2) {
    79    char *s;
    80    int i, x, n;
    81  
    82    s = str;
    83    while (*s != '\0') {
    84      s++;
    85    }
    86    x = s - str; /* strの文字数 */
    87    s = str;
    88    n=0; /*入れ替えを行った回数 */
    89    for(i=0;i<x+1;i++) {
    90      if (*(s+i)== c1) {
    91        *(s+i) = c2;
    92        n++;
    93      }
    94    }
    95    return n;
    96  }
    97  
    98  void parse_line(char *line) {
    99    if (*line == '%') {
   100      exec_command(line[1], line[2], &line[3]);
   101    } else {
   102      if (profile_data_nitems < 10000 ) {
   103        new_profile(line) ;
   104      }
   105    }
   106  }
   107  
   108  void exec_command(char cmd1, char cmd2, char *param) {
   109  
   110    if(cmd2 == ' ' || cmd2 == '\0') {
   111      /*************************** 1文字コマンド ****************************/
   112      switch (cmd1) {
   113      case 'Q' :
   114      case 'q' : cmd_quit();  break;
   115      case 'C' : 
   116      case 'c' : cmd_check(); break;
   117      case 'P' : 
   118      case 'p' : cmd_print(atoi(param)); break;
   119      case 'R' : 
   120      case 'r' : cmd_read(param);  break;
   121      case 'W' : 
   122      case 'w' : cmd_write(param); break;
   123      case 'F' : cmd_find(param); break;
   124      case 'f' : cmd_find2(param);  break;
   125      case 'S' : cmd_bsort(atoi(param));  break;
   126      case 's' : cmd_qsort(atoi(param));  break;
   127      case 'M' : 
   128      case 'm' : cmd_modify(atoi(param)); break;
   129      case 'D' : 
   130      case 'd' : cmd_delete(atoi(param)); break;
   131      default:
   132        fprintf(stderr, "%%%c command is undefined \n", cmd1);
   133        break;
   134      }
   135      /*******************************************************************/
   136    } else
   137      /************************** 2文字コマンド ****************************/
   138      switch (cmd1) {
   139      case 'T' :
   140      case 't' :
   141        switch (cmd2) {
   142        case 'R' : 
   143        case 'r' : cmd_read(param+1); break;
   144        default:
   145          fprintf(stderr, "%%%c%c command is undefined \n", cmd1, cmd2);
   146        break;
   147        } break;      
   148      case 'B' :
   149      case 'b' :
   150        switch (cmd2) {
   151        case 'R' : 
   152        case 'r' : cmd_Bread(param+1); break;
   153        case 'W' : 
   154        case 'w' : cmd_Bwrite(param+1); break;
   155        case 'S' : 
   156        case 's' : cmd_bsort(atoi(param+1));  break;  
   157        default:
   158          fprintf(stderr, "%%%c%c command is undefined \n", cmd1, cmd2);
   159          break;  
   160        } break;
   161      case 'Q' :
   162      case 'q' :
   163        switch (cmd2) {
   164        case 'S' : 
   165        case 's' : cmd_qsort(atoi(param+1)); break;
   166        default:
   167          fprintf(stderr, "%%%c%c command is undefined \n", cmd1, cmd2);
   168        break; 
   169        } break;
   170      default:
   171        fprintf(stderr, "%%%c%c command is undefined \n", cmd1, cmd2);
   172        break;
   173      }   
   174      /*******************************************************************/
   175  }
   176    
   177  void new_profile(char *line) {
   178      
   179    char *ret[5];
   180    char *ret2[3];
   181    int cnt1, cnt2;
   182    
   183    cnt1 = split(line, ret, ',', MAX);
   184    if (cnt1 == 5){
   185      if (strlen(ret[0]) > 8) {
   186        fprintf(stderr,"ID max digits are 8\n");
   187      }
   188      PDS[PDN].id = atoi(ret[0]);
   189      strncpy(PDS[PDN].name, ret[1], MAX_STR_LEN);
   190      PDS[PDN].name[MAX_STR_LEN] = '\0';
   191      cnt2 = split(ret[2], ret2, '-', 3);
   192      if (cnt2 == 3) { 
   193        PDS[PDN].birthday.y = atoi(ret2[0]);
   194        PDS[PDN].birthday.m = atoi(ret2[1]);
   195        PDS[PDN].birthday.d = atoi(ret2[2]);
   196        if (PDS[PDN].birthday.y > 9999 ||
   197            PDS[PDN].birthday.y < 0    ||
   198            PDS[PDN].birthday.m > 12   ||
   199            PDS[PDN].birthday.m < 1    ||
   200            PDS[PDN].birthday.d < 1    ||
   201            PDS[PDN].birthday.d > 31) {
   202          fprintf(stderr,"inputed birthday data is inappropriate\n");
   203          cnt2 = 0;
   204        }
   205      } else if (cnt2 == 2 || cnt2 == 1) {
   206        fprintf(stderr,"inputed birthday data is inappropriate\n");
   207      } 
   208      strncpy(PDS[PDN].home, ret[3], MAX_STR_LEN);
   209      PDS[PDN].home[MAX_STR_LEN] ='\0';
   210      PDS[PDN].comment = (char*)malloc(sizeof(char*)*(strlen(ret[4])+1));
   211      strcpy(PDS[PDN].comment, ret[4]);
   212      profile_data_nitems++;
   213      if (cnt2 != 3 || strlen(ret[0]) > 8) {
   214        profile_data_nitems--;
   215        if (cnt2 == 2 || cnt2 == 1 ) { 
   216        fprintf(stderr,"correct birthday form example: 1000-10-10\n");
   217        }
   218      }
   219    } else {
   220      fprintf(stderr,"error: this input is wrong form\n");
   221      fprintf(stderr,"correct form : (ID),(name),(birthday),(home),(comment)\n");
   222    }
   223  }
   224  
   225  int split(char *str, char *ret[], char sep, int max)
   226  {
   227    int cnt = 0;
   228  
   229    *(ret + (cnt++)) = str;
   230  
   231    while (*str && cnt < max) {
   232      if (*str == sep){
   233        *str = '\0';
   234        *(ret + (cnt++)) = str + 1;
   235      }
   236      str++;
   237    }
   238    
   239    return cnt; 
   240  }
   241  
   242  void cmd_quit() {
   243    int c;
   244    if (save == 0) {
   245      printf("you don't save date\n");
   246      printf("May I end this program? y or n\n");
   247      while((c = getchar()) == 'y' ||(c = getchar()) == 'n'){
   248        if (c = 'y') {
   249        exit(0);
   250        } else if (c = 'n') {
   251      } else
   252        printf("inputed character is wrong\n");
   253      }
   254    } else 
   255      exit(0);
   256  
   257      
   258  }
   259  
   260  void cmd_check() {
   261    printf("%d profile(s)\n",profile_data_nitems);
   262  }
   263  
   264  void cmd_print(int n) {
   265    int i;
   266    if (n > 0) {
   267      if ( n > profile_data_nitems ) {
   268        n = profile_data_nitems;
   269      }
   270      for( i = 0 ; i < n ; i++) { 
   271        print_profile(PDS + i);
   272      }
   273    } else if (n == 0) {
   274      for( i = 0 ; i < profile_data_nitems ; i++) { 
   275        print_profile(PDS + i);
   276      }
   277    } else if (n < 0 ) {
   278      if ( (-1) * n > profile_data_nitems) {
   279        n = (-1) * profile_data_nitems;
   280      } 
   281      for( i = profile_data_nitems + n ; i < profile_data_nitems ; i++) { 
   282        print_profile(PDS + i);
   283      }
   284    }
   285  }
   286  
   287  void cmd_read(char *file) {
   288    FILE *fp;
   289    char line[MAX_LINE_LEN + 1];
   290  
   291    fp = fopen(file, "r");
   292  
   293    if (fp == NULL) {
   294      fprintf(stderr,"Could not open file: $s\n", file);
   295    }
   296    while (get_line(fp, line)) {
   297      parse_line(line);
   298    }
   299  
   300    fclose(fp);
   301  }  
   302  
   303  void cmd_write(char *file) {
   304    int i;
   305    FILE *fp;
   306  
   307    fp = fopen(file, "w");
   308  
   309    if (fp == NULL) {
   310      fprintf(stderr,"Could not open file: $s\n", file);
   311    }
   312    for (i = 0; i < profile_data_nitems; i++) {
   313      fprint_profile_csv(fp, PDS+i);
   314    }
   315    fclose(fp);
   316  
   317    if (save == 0) {
   318      save = 1;
   319    }
   320  } 
   321  
   322  void cmd_find(char *word) {
   323    int i;
   324    for (i = 0; i < profile_data_nitems; i++) {
   325      if (comparsion(word, i) == 1) {
   326        print_profile(&profile_data_store[i]);
   327      }
   328    }
   329  } 
   330  
   331  void cmd_bsort(int n) {
   332    if(1 <= n || n <= 5) {
   333    b_sort(PDS, 0, profile_data_nitems - 1, n);
   334    } else 
   335      fprintf(stderr,"this argument is wrong. plese input 1,2,3,4,5\n");
   336  } 
   337  
   338  void cmd_qsort(int n) {
   339    if(1 <= n || n <= 5) {
   340    q_sort(PDS, 0, profile_data_nitems - 1, n);
   341    } else 
   342      fprintf(stderr,"this argument is wrong. plese input 1,2,3,4,5\n");
   343  } 
   344  
   345  void cmd_modify(int n) {
   346    if (n-1 < PDN && n > 0) {    
   347    char line[MAX_LINE_LEN + 1];
   348    get_line(stdin, line);
   349    
   350    char *ret[5];
   351    char *ret2[3];
   352    int cnt1, cnt2;
   353    
   354    cnt1 = split(line, ret, ',', MAX);
   355    if (cnt1 == 5){
   356      if (strlen(ret[0]) > 8) {
   357        fprintf(stderr,"ID max digits are 8\n");
   358      }
   359      PDS[n-1].id = atoi(ret[0]);
   360      strncpy(PDS[n-1].name, ret[1], MAX_STR_LEN);
   361      PDS[n-1].name[MAX_STR_LEN] = '\0';
   362      cnt2 = split(ret[2], ret2, '-', 3);
   363      if (cnt2 == 3) { 
   364        PDS[n-1].birthday.y = atoi(ret2[0]);
   365        PDS[n-1].birthday.m = atoi(ret2[1]);
   366        PDS[n-1].birthday.d = atoi(ret2[2]);
   367        if (PDS[n-1].birthday.y > 9999 ||
   368            PDS[n-1].birthday.y < 0    ||
   369            PDS[n-1].birthday.m > 12   ||
   370            PDS[n-1].birthday.m < 1    ||
   371            PDS[n-1].birthday.d < 1    ||
   372            PDS[n-1].birthday.d > 31) {
   373          fprintf(stderr,"inputed birthday data is inappropriate\n");
   374          cnt2 = 0;
   375        }
   376      } else if (cnt2 == 2 || cnt2 == 1) {
   377        fprintf(stderr,"inputed birthday data is inappropriate\n");
   378      } 
   379      strncpy(PDS[n-1].home, ret[3], MAX_STR_LEN);
   380      PDS[n-1].home[MAX_STR_LEN] ='\0';
   381      PDS[n-1].comment = (char*)malloc(sizeof(char)*strlen(ret[4])+1);
   382      strcpy(PDS[n-1].comment, ret[4]);
   383        if (cnt2 == 2 || cnt2 == 1 ) { 
   384        fprintf(stderr,"correct birthday form example: 1000-10-10\n");
   385        }
   386    } else {
   387      fprintf(stderr,"error: this input is wrong form\n");
   388      fprintf(stderr,"correct form : (ID),(name),(birthday),(home),(comment)\n");
   389    }
   390    } else
   391      fprintf(stderr,"error: your inputed argument is not correct\n");
   392  }
   393  
   394  void cmd_delete(int n) {
   395    int i;
   396    if (n-1 < PDN && n > 0) {    
   397    printf("Number %d date deleted\n", n);
   398    
   399    for(i = 0;i < PDN - (n-1);i++) {
   400      swap(&PDS[n-1], &PDS[n]);
   401      n++;
   402    }
   403    PDN--;
   404    } else
   405      fprintf(stderr,"error: your inputed argument is not correct\n");
   406  }
   407  
   408  void cmd_find2(char *word) {
   409    int i;
   410  
   411    for (i = 0; i < profile_data_nitems; i++) {
   412      if (comparsion(word, i) == 1) {
   413        print_profile(&profile_data_store[i]);
   414        printf("this data is number %d from start\n",i+1);
   415      }
   416    }
   417  } 
   418  
   419  void cmd_Bwrite(char *line) {
   420    int length; /*profile_date_storeのcommentの文字数を保存する変数*/
   421    int i;
   422    int p; /*commentを格納する配列を初期化することに使用する変数*/
   423    char a[100+1]; /*commentを格納する配列 100文字まで格納可能*/
   424    FILE *fpw = fopen(line, "wb");
   425  
   426    fwrite(&PDN, sizeof(int), 1, fpw);
   427    for(i=0;i<PDN;i++){
   428    length = strlen(PDS[i].comment);
   429    fwrite(&length, sizeof(int), 1, fpw);
   430    fwrite(&PDS[i], sizeof(PDS[i]), 1, fpw);
   431    for(p=0;p<101;p++) { /*配列を初期化*/
   432      a[p] = '\0';
   433    }
   434    if(length > 100) {
   435      length = 100;
   436    }
   437    strncpy(a, PDS[i].comment, length);
   438    a[100] = '\0';
   439    fwrite(&a, sizeof(char), 100+1, fpw);
   440    }
   441    
   442    fclose(fpw);
   443   
   444  } 
   445  
   446  void cmd_Bread(char *line) {
   447    int length; /*profile_date_storeのcommentの文字数を保存する変数*/
   448    int i;
   449    int bdn; /*読み込んだバイナリファイル中のデータ数*/
   450    FILE *fpr = fopen(line, "rb");
   451    char a[100+1]; /*commentを格納する配列 100文字を想定*/
   452    
   453    fread(&bdn, sizeof(int), 1, fpr);
   454    bdn = bdn + PDN;
   455    for(i=PDN;i<bdn;i++) {
   456    fread(&length, sizeof(int), 1, fpr);
   457    fread(&PDS[i], sizeof(PDS[i]), 1, fpr);
   458    PDS[i].comment = NULL; 
   459    fread(&a, sizeof(char), 100+1, fpr);
   460    PDS[i].comment = (char*)malloc(sizeof(char*)*length+1);
   461    strcpy(PDS[i].comment, a);
   462    PDN++;
   463    }
   464    fclose(fpr);
   465    
   466    if (save == 0) { /*Quitコマンドで使用*/
   467      save = 1;
   468    }
   469  }
   470  
   471  
   472  void print_profile(struct profile *p) {
   473    printf("Id    : %d\n",p->id);
   474    printf("Name  : %s\n",p->name);
   475    printf("Birth : %04d-%02d-%02d\n",p->birthday.y,p->birthday.m,p->birthday.d);
   476    printf("Addr  : %s\n",p->home);
   477    printf("Com.  : %s\n",p->comment);
   478    printf("\n");
   479  }
   480  
   481  void fprint_profile_csv(FILE *fp, struct profile *p) {
   482    fprintf(fp,"%d,",p->id);
   483    fprintf(fp,"%s,",p->name);
   484    fprintf(fp,"%d-%d-%d,",p->birthday.y,p->birthday.m,p->birthday.d);
   485    fprintf(fp,"%s,",p->home);
   486    fprintf(fp,"%s\n",p->comment);
   487      
   488  }
   489  
   490  char *date_to_string(char buf[], struct date *date)
   491  {
   492    sprintf(buf, "%04d-%02d-%02d", date->y, date->m, date->d);
   493    return buf;
   494  }
   495  
   496  int comparsion (char *word, int i) {
   497    char buf[8 + 1]; /* IDは8桁と制限しているため8+1としている*/
   498    char birthday_str[10 + 1]; /*birthdayは"-"を含めて10桁になるよう制限しているため*/
   499    
   500    sprintf(buf, "%d", PDS[i].id);
   501    if (strcmp(PDS[i].name, word)    == 0  ||
   502        strcmp(PDS[i].home, word)    == 0  ||
   503        strcmp(PDS[i].comment, word) == 0  ||
   504        strcmp(buf, word)         == 0  ||
   505        strcmp(date_to_string(birthday_str, &PDS[i].birthday), word) ==0) {
   506      return 1;
   507    } else 
   508      return 0;
   509  }
   510  
   511  int compare_profile(struct profile *p1, struct profile *p2, int column) {
   512    switch (column) {
   513    case 1:
   514      return p1->id - p2->id;
   515    case 2:
   516      return strcmp(p1->name, p2->name);
   517    case 3:
   518      if (p1->birthday.y != p2->birthday.y) return p1->birthday.y - p2->birthday.y;
   519      if (p1->birthday.m != p2->birthday.m) return p1->birthday.m - p2->birthday.m;
   520      return p1->birthday.d - p2->birthday.d;
   521    case 4:
   522      return strcmp(p1->home, p2->home);
   523    case 5:
   524      return strcmp(p1->comment, p2->comment);
   525    
   526      
   527    }
   528  }
   529  
   530  void b_sort(struct profile *p, int left, int right, int column) {
   531    int i, j;
   532  
   533    for (i = left; i <= right; i++) {
   534      for (j = left; j <= right - 1; j++) {
   535        if ((compare_profile(p+j, p+j+1, column)) > 0) {
   536            swap(&p[j], &p[j + 1]);
   537        }
   538      }
   539    }
   540  }
   541  
   542  void swap(struct profile *a, struct profile *b) {
   543    struct profile temp;
   544  
   545    temp = *a;
   546    *a = *b;
   547    *b = temp;
   548  }
   549  
   550  
   551  
   552  void q_sort(struct profile *p, int left, int right, int column) {
   553    int i,j;
   554    int pivot;
   555    pivot = (left + right) / 2;
   556    
   557    i = left;
   558    j = right;
   559  
   560    while(1) {
   561      while ((compare_profile(p+pivot, p+i, column)) > 0) {
   562        i++;
   563      }
   564      while ((compare_profile(p+j, p+pivot, column)) > 0) {
   565        j--;
   566      }
   567        if (i <= j) {
   568          break;
   569        }
   570      swap(&p[i], &p[j]);
   571      i++;
   572      j--;
   573    }
   574      q_sort(p,left, j, column);
   575      q_sort(p, i, right, column);
   576  }

\end{verbatim}
}
 
\end{document}
