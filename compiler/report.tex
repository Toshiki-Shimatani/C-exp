\documentclass[a4paper,11pt]{jarticle}
% ファイル先頭から\begin{document}までの内容(プレアンブル)については,
% 教員からの指示がない限り, { } の中を書き換えるだけでよい.

% ToDo: 提出要領に従って,適切な余白を設定する
\usepackage[top=25truemm,  bottom=30truemm,
            left=25truemm, right=25truemm]{geometry}


% ToDo: 提出要領に従って,適切なタイトル・サブタイトルを設定する.
\title{情報工学実験Cレポート \\
      コンパイラ実験}

% ToDo: 自分自身の氏名と学生番号に書き換える
\author{氏名: 島谷 隼生 (Shimatani, Toshiki) \\
        学生番号: 09428526}

% ToDo: 教員の指示に従って適切に書き換える
\date{出題日: 2018年12月06日 \\
      提出日: 2019年02月05日 \\
      締切日: 2019年02月05日 \\}  % 注:最後の\\は不要に見えるが必要.

% ToDo: 図を入れる場合,以下の1行を有効にする
%\usepackage{graphicx}
\usepackage{ascmac}

\begin{document}
\maketitle

% 目次つきの表紙ページにする場合はコメントを外す
%{\footnotesize \tableofcontents \newpage}

%%%%%%%%%%%%%%%%%%%%%%%%%%%%%%%%%%%%%%%%%%%%%%%%%%%%%%%%%%%%%%%%
\section{実験の目的}
%%%%%%%%%%%%%%%%%%%%%%%%%%%%%%%%%%%%%%%%%%%%%%%%%%%%%%%%%%%%%%%%

本実験では,情報系学科で学んだ3年間の総仕上げとしてコンパイラの作成を行うことを通じ,
プログラミング言語で書かれたプログラムと
アセンブリ言語の対応についてより深く理解し,木構造の取扱いについて習熟すること,
およびyacc.lexといったプログラムジェネレータを使用してプログラムを作成する経験を
積むことを目的とする.

%%%%%%%%%%%%%%%%%%%%%%%%%%%%%%%%%%%%%%%%%%%%%%%%%%%%%%%%%%%%%%%%
\section{作成した言語定義}
最終的に作成した言語の定義をBNFで記述する.以下がその言語定義である.
{\fontsize{10pt}{11pt} \selectfont
\begin{itembox}[l]{作成した言語定義(1/3}
\begin{eqnarray*}
  <program> &::=& <variable\_declarations> <function\_list> \\
  <function\_list> &::=& <function> <function\_list> \\
    &|& <function> \\
  <function\_list> &::=& <function> <function\_list> \\
  <function> &::=& <pre\_func> (<argument\_list>) \\&&{<variable\_declarations> <statement\_list>} \\
  &|& <pre\_func> () \\&& \{ <variable\_declarations>  <statement\_list> \} \\
<pre\_func> &::=& \verb|func| <\verb|IDENTIFIER|> \\
<argument\_list> &::=& <argument>\verb|,| <argument\_list> \\
                  &|& <argument>\\
<argument> &::=& \verb|define| <\verb|IDENTIFIER|>\\
             &|& \verb|array| <\verb|IDENTIFIER|>[] \\
             &|& \verb|array| <\verb|IDENTIFIER|> [<\verb|NUMBER|>][] \\
<variable\_declarations> &::=& <declaration> <variable\_declarations>  \\
\end{eqnarray*}
\end{itembox}
\begin{itembox}[l]{作成した言語定義(2/3)}
\begin{eqnarray*}
<declaration> &::=& \verb|define| <identifier\_list>\verb|;|    \\
                &|& \verb|array| <\verb|IDENTIFIER|> [<\verb|NUMBER|>]\verb|;|  \\ 
                &|& \verb|array| <\verb|IDENTIFIER|> [<\verb|NUMBER|>][<\verb|NUMBER|>]\verb|;| \\
<identifier\_list> &::=& <\verb|IDENTIFIER|>\verb|,| <identifier\_list> \\
                    &|& <\verb|IDENTIFIER|>\\
<statement\_list> &::=& <statement> <statement\_list>\\
                   &|& <statement>          \\
<statement> &::=& <assignment\_statement>    \\
              &|& <loop\_statement>\\
              &|& <selection\_statement>\\
              &|& <function\_call>\\
              &|& <break\_statement> \\
              &|& <\verb|IDENTIFIER|> <unary\_operator> \\
              &|& <unary\_operator> <\verb|IDENTIFIER|>\\
<assignment\_statement> &::=& <\verb|IDENTIFIER|> \verb|=| <arithmetic\_expression>\verb|;| \\
                         &|& <array\_reference> \verb|=| <arithmetic\_expression>\verb|;|\\
<arithmetic\_expression> &::=& <arithmetic\_expression> <additive\_operator> \\&&<multiplicative\_expression> \\
                          &|& <multiplicative\_expression> \\
<multiplicative\_expression> &::=& <multiplicative\_expression>\\&&<multiplicative\_operator> <primary\_expression> \\
                              &|& <primary\_expression>        \\
primary\_expression &::=& <variable>\\
&|& (<arithmetic\_expression>)\\
<additive\_operator> &::=& \verb|+| \\
&|& \verb|-| \\
<multiplicative\_operator> &::=& \verb|*| \\
&|& \verb|/|\\
&|& \verb|%| \\
<unary\_operator> &::=& \verb|++| \\
&|& \verb|--| \\
<variable> &::=& \verb|IDENTIFIER| \\
&|& \verb|NUMBER|\\
&|& <array\_reference> \\
&|& \verb|IDENTIFIER| <unary\_operator>\\
&|& <unary\_operator> \verb|IDENTIFIER|\\
<array\_reference> &::=& \verb|IDENTIFIER| [<variable>] \\
&|& \verb|IDENTIFIER| [<variable>][<variable>]\\
<loop\_statement> &::=& \verb|while|(<expression>){<statement\_list>}\\
&|& \verb|WHILE|(<expression>)<statement>\\
&|& \verb|FOR|\\&&(<for\_initial><for\_expression><for\_update>)\\&&{<statement\_list>}\\
&|& \verb|FOR|\\&&(<for\_initial><for\_expression><for\_update>)\\&&<statement>\\
\end{eqnarray*}
\end{itembox}
\begin{itembox}[l]{作成した言語定義(3/3)}
\begin{eqnarray*}
<for\_initial> &::=& <assignment\_statement>\\
&|& \verb|;|\\
<for\_expression> &::=& <expression> \verb|;|\\ 
&|& \verb|;|\\
<for\_update> &::=& <\verb|IDENTIFIER|> \verb|=| <arithmetic\_expression>\\
&|& <array\_reference> = <arithmetic\_expression> \\
&|& \verb|IDENTIFIER| <unary\_operator>\\
&|& <unary\_operator> <\verb|IDENTIFIER|> \\
&|& (何もないことを定義)\\
<selection\_statement> &::=& <if\_statement> \\
&|& <if\_statement> <else\_statement>\\
<if\_statement> &::=& \verb|IF| (<expression>){<statement\_list>}\\
&|& \verb|IF|(<expression>)<statement>\\
<else\_statement> &::=& \verb|ELSE|{<statement\_list>}\\
&|& \verb|ELSE| <statement> \\
<break\_statement> &::=& \verb|BREAK| \verb|;|\\
<expression> &::=& <arithmetic\_expression> <comparison\_operator>\\&& <arithmetic\_expression> \\
<comparison\_operator> &::=& \verb|==| \\
&|& \verb|<|\\
&|& \verb|>|\\
&|& \verb|≦|\\
&|& \verb|≧|\\
<function\_call> &::=& \verb|FUNCCALL| \verb|IDENTIFIER|()\verb|;|\\
&|& \verb|FUNCCALL| \verb|IDENTIFIER|(<parameter\_list>) \verb|;|\\
<parameter\_list> &::=& <arithmetic\_expression> \verb|,| <parameter\_list>\\
&|& <arithmetic\_expression> \\
\end{eqnarray*}
\end{itembox}

%%%%%%%%%%%%%%%%%%%%%%%%%%%%%%%%%%%%%%%%%%%%%%%%%%%%%%%%%%%%%%%%
%%%%%%%%%%%%%%%%%%%%%%%%%%%%%%%%%%%%%%%%%%%%%%%%%%%%%%%%%%%%%%%%
\section{言語定義で受理されるプログラムの例}
%%%%%%%%%%%%%%%%%%%%%%%%%%%%%%%%%%%%%%%%%%%%%%%%%%%%%%%%%%%%%%%%

%%%%%%%%%%%%%%%%%%%%%%%%%%%%%%%%%%%%%%%%%%%%%%%%%%%%%%%%%%%%%%%%
\section{コード生成の概略}
%%%%%%%%%%%%%%%%%%%%%%%%%%%%%%%%%%%%%%%%%%%%%%%%%%%%%%%%%%%%%%%%


%%%%%%%%%%%%%%%%%%%%%%%%%%%%%%%%%%%%%%%%%%%%%%%%%%%%%%%%%%%%%%%%
\section{コンパイラ作成過程で工夫した点}
%%%%%%%%%%%%%%%%%%%%%%%%%%%%%%%%%%%%%%%%%%%%%%%%%%%%%%%%%%%%%%%%

%%%%%%%%%%%%%%%%%%%%%%%%%%%%%%%%%%%%%%%%%%%%%%%%%%%%%%%%%%%%%%%%
\section{最終課題を解くために書いたプログラムの概要}
%%%%%%%%%%%%%%%%%%%%%%%%%%%%%%%%%%%%%%%%%%%%%%%%%%%%%%%%%%%%%%%%


%%%%%%%%%%%%%%%%%%%%%%%%%%%%%%%%%%%%%%%%%%%%%%%%%%%%%%%%%%%%%%%
\section{最終課題の実行結果}
%%%%%%%%%%%%%%%%%%%%%%%%%%%%%%%%%%%%%%%%%%%%%%%%%%%%%%%%%%%%%%%%

%%%%%%%%%%%%%%%%%%%%%%%%%%%%%%%%%%%%%%%%%%%%%%%%%%%%%%%%%%%%%%%
\section{最終課題の考察}
%%%%%%%%%%%%%%%%%%%%%%%%%%%%%%%%%%%%%%%%%%%%%%%%%%%%%%%%%%%%%%%%

%%%%%%%%%%%%%%%%%%%%%%%%%%%%%%%%%%%%%%%%%%%%%%%%%%%%%%%%%%%%%%%%
\section{ソースプログラムのある場所}
%%%%%%%%%%%%%%%%%%%%%%%%%%%%%%%%%%%%%%%%%%%%%%%%%%%%%%%%%%%%%%%%

作成したソースプログラ以下の場所に保存してある..

%%%%%%%%%%%%%%%%%%%%%%%%%%%%%%%%%%%%%%%%%%%%%%%%%%%%%%%%%%%%%%%%
\section{プログラムリスト}
%%%%%%%%%%%%%%%%%%%%%%%%%%%%%%%%%%%%%%%%%%%%%%%%%%%%%%%%%%%%%%%%
{\fontsize{10pt}{11pt} \selectfont
\begin{verbatim}

\end{verbatim}
}

\begin{verbatim}
<program> ::= <variable_declarations> <function_list> 
            | <function_list>
<function_list> ::= <function> <function_list> 
                  | <function>
<function> ::= <pre_func> (<argument_list>) {<variable_declarations> <statement_list>}
             | <pre_func> () {<variable_declarations> <statement_list>}
<pre_func> ::= func <IDENTIFIER>
<argument_list> ::= <argument>, <argument_list> 
                  | <argument>
<argument> ::= define <IDENTIFIER>
             | array <IDENTIFIER>[] 
             | array <IDENTIFIER> [<NUMBER>][] 
<variable_declarations> ::= <declaration> <variable_declarations>  
<declaration> ::= define <identifier_list>;    
                | array <IDENTIFIER> [<NUMBER>];   
                | array <IDENTIFIER> [<NUMBER>][<NUMBER>]; 
<identifier_list> ::= <IDENTIFIER>, <identifier_list> 
                    | <IDENTIFIER>
<statement_list> ::= <statement> <statement_list>
                   | <statement>          
<statement> ::= <assignment_statement>    
              | <loop_statement>
              | <selection_statement>
              | <function_call>
              | <break_statement> 
              | <IDENTIFIER> <unary_operator> 
              | <unary_operator> <IDENTIFIER>
<assignment_statement> ::= <IDENTIFIER> = <arithmetic_expression>; 
                         | <array_reference> = <arithmetic_expression>;
<arithmetic_expression> ::= <arithmetic_expression> <additive_operator> <multiplicative_expression> 
                          | <multiplicative_expression> 
<multiplicative_expression> ::= <multiplicative_expression> <multiplicative_operator> <primary_expression> 
                              | <primary_expression>        
primary_expression : variable
| PAREN_L arithmetic_expression PAREN_R
additive_operator : ADDITION 
| SUBTRACTION {$$ = AST_SUB;}
multiplicative_operator : MULTIPLICATION 
| DIVISION
| MODULUS 
unary_operator : INCREMENT 
| DECREMENT 
variable : IDENTIFIER 
| NUMBER
| array_reference 
| IDENTIFIER unary_operator
| unary_operator IDENTIFIER
array_reference : IDENTIFIER BRACKET_L variable BRACKET_R 
| IDENTIFIER BRACKET_L variable BRACKET_R BRACKET_L variable BRACKET_R 
loop_statement : WHILE PAREN_L expression PAREN_R BRACE_L statement_list BRACE_R 
| WHILE PAREN_L expression PAREN_R statement
| FOR PAREN_L for_initial for_expression for_update PAREN_R BRACE_L statement_list BRACE_R
| FOR PAREN_L for_initial for_expression for_update PAREN_R statement
for_initial : assignment_statement



\end{verbatim}

\end{document}
