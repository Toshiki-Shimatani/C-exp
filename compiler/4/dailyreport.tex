\documentclass[a4j]{jarticle}
\usepackage{enumitem}
\textheight 25cm
\topmargin -1.5cm

\begin{document}
\thispagestyle{empty}

\begin{flushright}
グループ番号: B3
\end{flushright}

\begin{center}
{\LARGE 作業報告書}
\end{center}

\subsection*{作業日}

2018年01月24日


\subsection*{作業者氏名}

\begin{itemize}
  \item メンバー1: 島谷 隼生
  \item メンバー2: 西川 晃生
  \item メンバー3: 加藤 大地
\end{itemize}

\subsection*{本日の作業目標}

%ここに本日の作業目標を書く
\begin{enumerate}
\item
コード生成の部分の作成


\end{enumerate}

\subsection*{本日の作業結果}

%ここに全体としてどこまで作業が進んだかを書く
最終課題1,2のコードを生成し,mapsで動作確認を行い,想定通りに動作することが確認できた.
以降,個々人で残りの最終課題のコード生成に取り組んだ,

\subsection*{特記事項(質問・意見・感想など)}

%何かあれば書く
順調に作業を進めることができています.個々人での作業の段階まで進みましたが,スタックの
確保の仕方を変更したり,算術式の生成の仕方を個人で改良した場合,最終課題1,2に使用されているため
グループで共有する必要があるのでしょうか?


\end{document}
