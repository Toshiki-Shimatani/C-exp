\documentclass[a4j]{jarticle}
\usepackage{enumitem}
\textheight 25cm
\topmargin -1.5cm

\begin{document}
\thispagestyle{empty}

\begin{flushright}
グループ番号: B3
\end{flushright}

\begin{center}
{\LARGE 作業報告書}
\end{center}

\subsection*{作業日}

2019年01月31日


\subsection*{作業者氏名}

\begin{itemize}
  \item メンバー1: 島谷 隼生

\end{itemize}

\subsection*{本日の作業目標}

%ここに本日の作業目標を書く
\begin{enumerate}
\item
コード生成の部分の作成


\end{enumerate}

\subsection*{本日の作業結果}

%ここに全体としてどこまで作業が進んだかを書く
個々人の作業の段階に移行したので,スタック領域の使用方法の見直しを行った.
前回では,プログラムを主に記述していた人の考えやすいように設計を行ったが,
MIPSの規約からははずれるものであったため,MIPS規約に則った形になるように作業を行っていた.
結果として,完成まではたどり着けなかったが,道筋を立てることはできた.

その他のコード生成については来週の試問日までに取り組む予定である.

\subsection*{特記事項(質問・意見・感想など)}

%何かあれば書く
特にありません.


\end{document}
